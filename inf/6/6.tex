\documentclass[12pt]
{article}
{\Large}
\usepackage[utf8]{inputenc}
\usepackage[T1]{fontenc}
\usepackage[russian]{babel}
\usepackage{geometry}
\usepackage{graphicx}
\graphicspath{ {im/} }
\renewcommand\footnoterule{\vspace*{0.3cm}\hrule width 6.8cm\vspace*{0.3cm}}
\geometry{verbose,a4paper,tmargin=2cm,bmargin=2cm,lmargin=1.5cm,rmargin=1.5cm}
\begin{document}
\begin{minipage}{0.43\textwidth}
ду которыми \(r_{0} \approx 3{\dot{A}}\) \footnote[*)]{*) Для кристаала \(NaCl\) это расстояние равно \(2.814{\dot{A}}\)}. Предположим,что деформация остается упругой вплоть до деформации, соответствующей разрыву; иначе говоря,  большей упругой деформации соответствует напряжение, равное пределу прочности. Но максимальной упругой деформации приблизительно соответствует максимальное значение силы межатомного притяжения (см. рис. 1). \\
\hspace*{16} Опыты с самыми прочными кристаллами показали, что их максимальная относительная упругая деформация \(\epsilon_{max}\) \footnote[**)]{**)Относительная деформация \(\epsilon\) рпи растяжении равна отношению абсолютной деформации тела к длине этого тела в нормальном состоянии} перед разрушением обычно не превышает 10-20\%. Положим \(\epsilon_{max} = \frac{1}{6} \approx 17\%\).
Этой относительной деформации соответствует смещение этомов от положения равновесия на расстояние \(\Delta r = \epsilon r_{0} = \frac{1}{6}3{\dot{A}} = 0.5{\dot{A}}\).
Таким образом, при подсчете сил межатомного притяжения для рассматриваемой модели кристалла за расстояние между нонами следует брать величину \\ \(r = r_{0} + \Delta r = 3{\dot{A}} + 0.5{\dot{A}} = 3.5{\dot{A}}\).\\
\hspace*{16} Если учесть, что заряд каждого иона по величине равен заряду электрона, то есть \(q = 1.6*10^{-19}\) \textit{к}, то максимальное значение силы притяжения между двумя атомами будет равно
\begin{center}\(F_{max} = \frac{1}{4\pi \epsilon_{0}} \frac{q^2}{r^2} = 9*10^9 \frac{(1.6*10^{-19})^2}{(3.5*10^{-10})^2} \approx\) 
\end{center}
\begin{flushright}
    \(\approx 2*10^{-9}\) (\textit{н}).
\end{flushright}
Таково по порядку величины значение единичной силы межатомной связи.
\begin{flushleft}
    \textbf{Прочность кристалла}
\end{flushleft}
Оценим примерное число атомов приходящихся на единицу поверхности разрыва кристалла
\end{minipage}
\hfill
\begin{minipage}{0.43\textwidth}
\hspace*{16}Диаметр иона равен приблизительно расстоянию между соседними ионами. Мы считали это расстояние равным \(3 \dot{A}\), тогда число атомов на каждом квадратном метре поверхности разрыва кристалла
\begin{center}
\(N_{\textit{ат}} \sim \frac{1}{(3*10^{-20})^2} \approx 10^{19} \hspace{3} (\frac{1}{\textit{м}^2})\).
\end{center}
\hspace*{16}В нашей модели кристалла число связей, проходящих через единицу площади, равно числу атомов \(N_{\textit{св}} = N_{\textit{ат}}\), значит, \(N_{\textit{св}} \approx 10^{19} \hspace{3} \textit{м}^{-2}\).\\
\hspace*{16}Теперь можно оценить теорети-
ческую величину предела прочности
кристаллов:
\begin{center}
\(\delta \approx 2*10^{10} \hspace{3} \textit{н}/\textit{м}^2\).
\end{center}

\begin{flushleft}
\textbf{Оценка величины модуля упругости}\end{flushleft}
Если известны значения единичной межатомной связи и, следовательно, предела прочности кристаллов, то можно оценить величину модуля упругости.

\hspace*{16}По закону Гука в пределах упругой деформации напряжение пропорционально растяжению. Коэффициент пропорциональности между величиной деформации \(\epsilon\) и напряжением \(\delta\) (модуль упругости)
\begin{center}
    \(E = \frac{\delta}{\epsilon}\).
\end{center}
\hspace*{16}Так как величина прочности по нашей оценке
\begin{center}
\(\delta \approx 2*10^{10} \hspace{3} \textit{н}/\textit{м}^2\),
\end{center}
а максимальная упругая деформация \(\epsilon_{max} \approx \frac{1}{6}\), то модуль упругости
\begin{center}
\(E = \frac{2*10^{10}*6}{1} \approx 10^{11}(\frac{\textit{н}}{\textit{м}^2})\).
\end{center}
\hspace*{16}Результат расчета по порядку величины соответствует экспериментальным данным, Например, модуль упругости стали \(2*10^{11} \hspace{3} \textit{н}/\textit{м}^2\), алюминия \(0.7*10^{11} \hspace{3} \textit{н}/\textit{м}^2\), каменной соли -- \(0.4*10^{11} \hspace{3} \textit{н}/\textit{м}^2\).

\end{minipage}

\begin{page}
\begin{flushleft}
\includegraphics{1.png} \\
Рисунок 1
\end{flushleft}
\begin{flushright}

    \includegraphics[scale=0.5]{2.png} \\
Рисунок 2 \\ 
\includegraphics[angle=160]{3.png} \\
Рисунок 3 \\ 

\end{flushright}
\begin{enumerate}
    \item 
\begin{tabular}{|l|l|l|p{3.5cm}|}
\hline
123 & 123 & 132 & 132 \\ 
132 & 123 & 123 & 123 \\
\hline
\end{tabular}
    \item
\begin{center}  
\begin{tabular}{|l|l|l||l|p{6cm}|}
\hline
123 & 123 & 132 & 132 & 132 \\ 
132 & 123 & 123 & 123 & 132 \\
\hline
\end{tabular}
\end{center}
    \item 
\begin{flushright}     
\begin{tabular}{|l|l|||||||||||||l|}
\hline
123 & 123 & 132 \\ 
132 & 123 & 123 \\
\hline
\end{tabular}
\end{flushright}
\end{enumerate}

\begin{tabular}{|l|l|}
\hline
123  123 132 &  \\ 
 & 123 123 \\
\hline
\end{tabular} \\


\begin{table}[b]
\caption{\label{tab:ььь} перемещённая вниз}
\begin{center}
\begin{tabular}{|c|c|c|}
\hline
\multirow{Можно и так} & \multicolumn{2}{c|}{что-то} \\
\cline{2-3}
            & Интересно & вау \\
\cline{2-3}
            & очень & 1 \\
\hline
\end{tabular}
\end{center}
\end{table}

\end{page}
\end{document}
